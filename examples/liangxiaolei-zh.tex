\documentclass[11pt,a4paper,sans]{moderncv}   % 

% moderncv 主题
\moderncvstyle{classic}                        % 选项参数是 ‘casual’, ‘classic’, ‘oldstyle’ 和 ’banking’
\moderncvcolor{blue}                          % 
%\nopagenumbers{}                             % 消除注释以取消自动页码生成功能

% 字符编码
\usepackage[utf8]{inputenc}                   % 替换你正在使用的编码
\usepackage{CJKutf8}

% 调整页面
\usepackage[scale=0.89]{geometry}
%\setlength{\hintscolumnwidth}{3cm}           % 如果你希望改变日期栏的宽度

% 个人信息
\firstname{\qquad梁晓雷}
\familyname{}
\title{}                     % 可选项、如不需要可删除本行
\address{湖北武汉--中国地质大学(武汉)--东区8号楼}{邮编:430074}
\phone[mobile]{+86 18986135762}              % 可选项、如不需要可删除本行
\email{18986135762@163.com}                    % 可选项、如不需要可删除本行
\homepage{blog.csdn.net/rainloving}                  % 可选项、如不需要可删除本行
%----------------------------------------------------------------------------------
%            内容
%----------------------------------------------------------------------------------
\begin{document}
\begin{CJK}{UTF8}{gbsn}                       % 详情参阅CJK文件包
\maketitle
 
\section{教育背景}
\cventry{2013--至今}{硕士(保研)}{中国地质大学(武汉)计算机学院}{计算机科学与技术}{}{}  % 第3到第6编码可留白
\cventry{2009--2013}{本科}{中国地质大学(武汉)计算机学院}{网络工程}{}{}

\section{求职意向}
\cventry{期望职位}{研发工程师}{}{}{}{}

\section{技能}
\cventry{IT技能}{能熟练运用C/C++、Python语言}{}{}{}{}
\cventry{}{熟悉基础的数据结构、算法、网络编程}{}{}{}{}
\cventry{}{了解Linux开发平台、SQL}{}{}{}{}
\cventry{}{有软件结构设计和需求分析的能力}{}{}{}{}
\cventry{}{能深入理解面向对象的思想,并能熟练地应用于具体的软件设计开发工作中}{}{}{}{}{}
\cventry{英语}{CET-6}{}{}{}{}
\cventry{GitHub}{\href{https://github.com/wutongjie23hao}{https://github.com/wutongjie23hao}}{}{}{}{}

\section{出版物}
\cventry{2014.6}{\textbf{Liang X}\textnormal{,Zeng S, High-gain wide-band dual-circular-polarised antenna design using dynamic constrained multi-objective evolutionary algorithm[J].International Journal of Innovative Computing and Applications, 2014, 6(1): 25-32}}{}{}{}{}

\section{项目经历}
\subsection{研究项目}
\renewcommand{\baselinestretch}{1.2}

\cventry{2014-至今}
{演化天线设计软件平台的设计和实现}
{Python,并行计算,操作系统}
{研究项目/国家自然基金项目}{}
{本项目开发一个软件平台,集成天线模型模块和演化算法模块,搭建演化天线优化设计平台,使用户在指定要使用的演化算法和天线模型后,平台自动地用指定的演化算法对指定的天线模型进行优化。我主要负责FEKO软件(电磁仿真软件)接口、RFID天线模型模版的代码实现和维护、框架代码的优化重构。}
\vspace*{0.2\baselineskip}

\cventry{2013-2014}
{利用动态约束多目标演化算法设计高增益宽频带双极化天线}
{python,分布式并行计算,云计算}
{研究项目/国家自然基金项目}{}
{本项目利用动态约束多目标演化算法(一种优化算法)对一款高增益高频双极化天线进行了设计和优化。天线建模模块以及部分算法模块的代码实现。该\textbf{\href{http://www.inderscienceonline.com/doi/abs/10.1504/IJICA.2014.064217?journalCode=ijica}{研究成果}}已发表在International Journal of Innovative Computing and Applications期刊上。}
\vspace*{0.2\baselineskip}

\cventry{2012}
{用演化算法对折线形状的RFID天线进行设计}
{Linux,C++,分布式并行计算}
{研究项目/国家自然基金项目}{}
{本项目用差分演化算法(一种优化算法)对一种折线形状的RFID天线进行了优化设计,使天线性能得到提高。其中我主要负责天线建模部分的代码实现。}
\vspace*{0.2\baselineskip}

\subsection{独立项目}
\cventry{2014}
{土木实验室信息管理系统}
{python,PySide,SQLite}
{独立项目/开源项目}{}
{本项目目的是编写一款软件,实现对土木实验室的实验仪器进行可视化管理,涉及仪器信息和仪器故障信息的数据统计、展示,并可导出excel格式的数据报表。由我和工程学院的一名同学完成,其中,我主要负责软件的代码实现(Source Code available on Github)。}
\vspace*{0.2\baselineskip}

\cventry{2014}
{个人文档管理系统}
{C++,Qt,SQLite}
{独立项目/开源项目}{}
{本项目实现了一个软件平台,包括两个管理系统:论文管理系统、文献管理系统,使个人研究论文与相关的参考文献一目了然。由我和工程学院的一名同学完成,其中,我主要负责软件平台的代码实现(Source Code available on Github)。}
\vspace*{0.2\baselineskip}

\section{奖项}
\cventry{2012}{国家奖学金}{}{}{}{}

\section{校内经历}
\cventry{2011-2013}{中国地质大学(武汉)193091班 班长}{}{}{}{}

\section{个人兴趣}
\cvitem{音乐}{\small 技术之余,会听音乐放松}
\cvitem{读书}{\small 技术类、文学类和经济类书籍}
\cvitem{新技术}{\small 喜欢了解科技动态}
\cvitem{运动}{\small 篮球}

\section{个人评价}
\cvlistitem{性格开朗,与人和善,有合作精神}
\cvlistitem{有发现问题、分析问题、解决问题的能力}
\cvlistitem{能坚持不懈地去做一件事}

\renewcommand{\baselinestretch}{1.0} 
 
\closesection{}                   % needed to renewcommands 
\renewcommand{\listitemsymbol}{-} % change the symbol for lists 
\clearpage\end{CJK} 
\end{document} 
%% 文件结尾 `template-zh.tex'.
